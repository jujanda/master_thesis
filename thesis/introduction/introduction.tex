%=========================================================================================================================================
%=========================================================================================================================================
\chapter{Introduction} \label{chapter:Introduction}
%=========================================================================================================================================
%=========================================================================================================================================

%-----------------------------------------------------------------------------------------------------------------------------------------
\section{Motivation}
%-----------------------------------------------------------------------------------------------------------------------------------------
%history/purpose of tcp/ip \\
%inet usage changed, new requirements \\
%old model doesn't cut it anymore \\
%mainly content delivery now, instead of point to point \\
%way better model possible: ICN \\
%different approaches, focus on one: NDN \\
%currently in development \\
%proof todays application types work with NDN needed\\
%conversational services is one of those types \\


When the first iterations of the Internet were designed, the original idea was to enable point-to-point conversations between two entities in a network. Thus, the Internet Protocol (IP) was born. While it was a ground-breaking, unique architecture considering the requirements of that time, it is nowhere near perfect. Especially since the way we use the Internet has drastically changed over the last few decades. More and more people interact with it every day and many business and financial services are being moved online. Social media and multimedia applications and services now form the biggest bulk of Internet traffic. The number of connected devices is rising and shows no signs of stopping with an increasing amount of smaller household items and sensors becoming part of the network. Even though the original design was not intended for any of these uses, many clever over-the-top solutions and adaptations made them possible anyway. Solving today's problems with old technology is however becoming increasingly complex and error-prone. As described in \cite{ZEBJ10}: \\

\textit{ "Just as the telephone system would be a poor vehicle for the broadcast content distribution done by TV and radio, the Internet architecture is a poor match to its primary use today."} \\

%TODO add sources for other approaches in ICN from Paper Investigating the Performance of Pull-based Dynamic Adaptive Streaming in NDN ( [4], [5],[6], [7], and [8])

%TODO check if all programmed sources are even used. Prune uncited sources.

This is the reason researchers around the globe are currently working on new solutions that are better suited for today's usage and challenges. One branch of them is collectively referred to as Information-Centric Networking (ICN), where the focus lies on the transported information instead of the connection between two nodes. While there are many different approaches within ICN, this thesis only focuses on one of the more promising ones called Named Data Networking (NDN) \cite{ZEBJ10}. Unlike IP, which is always about sending a datagram from source to destination, NDN changes the semantics of network service from \textit{"delivering the packet to a given destination address"} to \textit{"fetching data identified by a given name"} \cite{ZABJ14}. This simple change in design alone potentially allows NDN to offer better communication and faster content distribution. Furthermore, NDN is designed to include built-in network control and security, multi-path forwarding and in-network storage, while also being self-regulating. \cite{ZEBJ10, ZABJ14}

Although all of this sounds very promising, the research behind NDN is still far from done. To surpass and eventually replace the current Internet architecture, NDN needs to be capable of running everything we enjoy today at least as well as current solutions, preferably better. Thus, for the last decade, scientists have designed and developed several program prototypes to showcase the advantages of NDN, but also to highlight many of the problems which are still open. 

%-----------------------------------------------------------------------------------------------------------------------------------------
\section{Thesis Goal}
%-----------------------------------------------------------------------------------------------------------------------------------------
In this thesis, we aim to further the development of NDN-RTC \cite{GuBu15}, a library designed to provide Real-Time Communications (RTC) within the experimental NDN architecture. It offers methods for producing and consuming different concurrent media streams and comes with a small basic client application. The library was published in 2015 but is still under active development and has many features openly declared as future work. One such feature is Adaptive Rate Control (ARC), which would allow live switching of media streams in different qualities, depending on the current network situation. The goal of this thesis is, therefore, to investigate, design, and implement ARC functionality into the open NDN-RTC library and evaluate the potential gain in performance under different scenarios.

%What is the point of this work? \\
%What does this want to achieve? \\

%-----------------------------------------------------------------------------------------------------------------------------------------
\section{Thesis Structure}
%-----------------------------------------------------------------------------------------------------------------------------------------
The rest of this thesis is structured as follows: 

%The rest of the paper is structured as follows: Section II goes into detail on how NDN aims to provide all these features, followed by current limitations and open research topics in section III. Sections IV and V showcase ways of experimenting with NDN in a scientific way, be it software based simulation or a hardware based testbed. 

Chapter 1 explains... \\
Chapter 2 describes... \\
Chapter 3 presents... \\
etc.
%TODO write chapter overview
%=========================================================================================================================================
%=========================================================================================================================================
% EOF
%=========================================================================================================================================
%=========================================================================================================================================
