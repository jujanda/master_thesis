%=========================================================================================================================================
%=========================================================================================================================================
\chapter{Background} \label{chapter:Background}
%=========================================================================================================================================
%=========================================================================================================================================



%-----------------------------------------------------------------------------------------------------------------------------------------
\section{Motivation}
%-----------------------------------------------------------------------------------------------------------------------------------------

%history/purpose of tcp/ip
%inet usage changed, new requirements
%old model doesn't cut it anymore
%mainly content delivery now, instead of point to point
%way better model possible: ICN
%different approaches, focus on one: NDN

When the first iterations of the Internet were designed, the original idea was to enable point-to-point conversations between two entities in a network. Thus, the Internet Protocol (IP) was born. While it was a ground-breaking, unique architecture considering the requirements of that time, it is nowhere near perfect. Especially since the way we use the Internet has drastically changed over the last few decades. More and more people interact with it every day and many business, bank and financial services are being moved online. Social media and multimedia applications and services now form the biggest bulk of Internet traffic.%also mention IoT
Even though the original Design was not intended for any of these uses, many clever over-the-top solutions and adaptations made them possible anyway. Solving today?s problems with old technology is however becoming increasingly complex and error prone. As described in \cite{ZEBJ10}: \\

\textit{ "Just as the telephone system would be a poor vehicle for the broadcast content distribution done by TV and radio, the Internet architecture is a poor match to its primary use today."} \\

This is the reason researches around the globe are currently working on new solutions which are better suited for today?s usage and challenges. One branch of them is collectively referred to as Information-Centric Networking (ICN). While there are many different approaches within ICN, this work only focuses on one of the more promising ones called Named Data Networking (NDN). Unlike IP, which is always about sending a datagram from source to destination, NDN changes the semantics of network service from "delivering the packet to a given destination address" to "fetching data identified by a given name" \cite{ZABJ14}. This simple change in design alone potentially allows NDN to offer better communication and faster content distribution. Furthermore, NDN is designed to include built-in network control and security, multi-path forwarding and in-network storage, while also being self-regulating. Additionally, while NDN would ideally replace the current IP architecture as a whole and start fresh as a clean-slate solution, it is designed as universal overlay, meaning it can run over anything (including IP), while also enabling IP to run over NDN. This would allow a relatively painless and incremental deployment process, once it is sufficiently researched. \cite{ZEBJ10, ZABJ14}
%The rest of the paper is structured as follows: Section II goes into detail on how NDN aims to provide all these features, followed by current limitations and open research topics in section III. Sections IV and V showcase ways of experimenting with NDN in a scientific way, be it software based simulation or a hardware based testbed.

%-----------------------------------------------------------------------------------------------------------------------------------------
\section{NDN Fundamentals}
%-----------------------------------------------------------------------------------------------------------------------------------------
hourglass modell + figure \\
interest \& data packets + figure \\
datagram structure + figure \\
Router components and workflow (PID, FIB, CS) + figure \\

%-----------------------------------------------------------------------------------------------------------------------------------------
\section{NDN Specifics}
%-----------------------------------------------------------------------------------------------------------------------------------------
Maybe short intro to the following subsections?

%-----------------------------------------------------------------------------------------------------------------------------------------
\subsection{Naming}
%-----------------------------------------------------------------------------------------------------------------------------------------

%-----------------------------------------------------------------------------------------------------------------------------------------
\subsection{Forwarding}
%-----------------------------------------------------------------------------------------------------------------------------------------

%-----------------------------------------------------------------------------------------------------------------------------------------
\subsection{Caching}
%-----------------------------------------------------------------------------------------------------------------------------------------

%-----------------------------------------------------------------------------------------------------------------------------------------
\subsection{Security}
%-----------------------------------------------------------------------------------------------------------------------------------------

%-----------------------------------------------------------------------------------------------------------------------------------------
\subsection{Mobility}
%-----------------------------------------------------------------------------------------------------------------------------------------


%-----------------------------------------------------------------------------------------------------------------------------------------
\section{NDN Drawbacks}
%-----------------------------------------------------------------------------------------------------------------------------------------
Current obstacles and problems \\
Current open questions in science \\

%=========================================================================================================================================
%=========================================================================================================================================
% EOF
%=========================================================================================================================================
%=========================================================================================================================================
